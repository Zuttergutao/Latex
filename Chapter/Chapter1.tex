\Chapter{第一章}

\section{二级标题}
This is the text in first paragraph. This is the text in first 
paragraph. This is the text in first paragraph. \par
This is the text in second paragraph. This is the text in second 
paragraph. This is the text in second paragraph.

This is another paragraph, contains some text to test the paragraph
interlining, paragraph indentation and some other features. Also, 
is easy to see how new paragraphs are defined by simply entering a 
double blank space.


\subsection{图}


\begin{figure}[!htp]
    \begin{center}
        \includegraphics[width=5cm]{fig/天津大学logo.jpeg}
    \end{center}
    \vspace{-2em}
    \caption{天津大学校徽.}
    \vspace{-1em}
    \addtocounter{figure}{-1}
    \renewcommand{\figurename}{Fig.}    
    \caption{the title of the sub-figure.} 

\end{figure}

\newpage

\subsection{表}

\begin{center}
    \begin{table}[!htp]
        \vspace{-2em}
        \caption{答辩信息.}
        \vspace{-1.5em}
        \addtocounter{table}{-1}
        \renewcommand{\tablename}{Table.}    
        \caption{the title of the Table.} 
        \vspace*{-1em}
        \song\xiaosi{
            \setlength{\tabcolsep}{2em}{
                \begin{tabular}{|c|c|c|c|}
                    \hline 
                    \textbf{答辩日期} & \multicolumn{3}{c}{~~~~~~年~~~~月~~~~日} \vline\\
                    \hline 
                    \textbf{答辩委员会} & \textbf{姓名}  & \textbf{职称}  & \textbf{工作单位} \\
                    \hline
                    \textbf{主席} & xx  & xxx  & xxxxxx \\
                    \hline
                    \multirow{2}{*}{\textbf{委员}} & xx  & xxx  & xxxxxx \\
                                \cline{2-4} 
                                & xx  & xxx  & xxxxxx \\
                    \hline
                \end{tabular} 
            }
    }
    \end{table}  
\end{center}

\newpage

\subsubsection{公式}

\begin{equation}
    SSE=\sum^{n}_{i=1}(x_{sim}-x_{exp})^2 \label{eq1}
\end{equation}

上述公式\ref{eq1}为平方误差之和

\subsubsection{脚注}

注释应采用文中编号加脚注的模式。脚注应采用阿拉伯数字上标,字体为Times New Roman,
分章节连续标号\footnote{脚注字号为五号,其余样式与正文相同。}。
