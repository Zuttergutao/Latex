% 设置字体
\setmainfont{Times New Roman}
\setCJKmainfont{SimSun} 

% 设置段落行间距
\renewcommand{\baselinestretch}{1.67}  %两倍行距
\setlength{\parindent}{2em} % 首行缩进
\setlength{\parskip}{0pt} % 段落前后间距0

% 设置页面、页边距
\geometry{paperwidth=210mm, paperheight=297mm}
\geometry{top=27.5mm,bottom=25.4mm, left=35.7mm, right=37.7mm}


\newcommand{\song}{\songti}    % 宋体
\newcommand{\fs}{\fangsong}        % 仿宋体
\newcommand{\kai}{\kaishu}      % 楷体
\newcommand{\hei}{\heiti}      % 黑体
\newcommand{\li}{\lishu}        % 隶书
\newcommand{\yihao}{\fontsize{26pt}{26pt}\selectfont}       % 一号, 单倍行距
\newcommand{\xiaoyi}{\fontsize{24pt}{24pt}\selectfont}      % 小一, 单倍行距
\newcommand{\erhao}{\fontsize{22pt}{1.25\baselineskip}\selectfont}       % 二号, 1.25倍行距
\newcommand{\xiaoer}{\fontsize{18pt}{18pt}\selectfont}      % 小二, 单倍行距
\newcommand{\sanhao}{\fontsize{16pt}{16pt}\selectfont}      % 三号, 单倍行距
\newcommand{\xiaosan}{\fontsize{15pt}{15pt}\selectfont}     % 小三, 单倍行距
\newcommand{\sihao}{\fontsize{14pt}{14pt}\selectfont}       % 四号, 单倍行距
\newcommand{\xiaosi}{\fontsize{12pt}{12pt}\selectfont}      % 小四, 单倍行距
\newcommand{\wuhao}{\fontsize{10.5pt}{10.5pt}\selectfont}   % 五号, 单倍行距
\newcommand{\xiaowu}{\fontsize{9pt}{9pt}\selectfont}        % 小五, 单倍行距
\newcommand{\Chapter}[1]{\fancypagestyle{plain}{\pagestyle{fancy}}\chapter{#1}} % 使用\Chapter代替\chapter 用以显示页眉
\makeatletter
\newcommand\dlmu[2][4cm]{\hskip1pt\underline{\hb@xt@ #1{\hss#2\hss}}\hskip3pt} %设置下划线长度
\makeatother
\renewcommand{\chaptermark}[1]{\markboth{第\,\thechapter\,章\,#1}{}} % 设置中文页眉样式
\renewcommand\arraystretch{1.3}   % 表格两倍行高

% 设置标题深度
\setcounter{secnumdepth}{4}
% 章标题设置(一级标题)
\titleformat{\chapter}{\centering\xiaosan\hei}{第\,\thechapter\,章}{1em}{}
\titlespacing{\chapter}{0pt}{30pt plus 6pt}{30pt plus 6pt}
% 二级标题
\titleformat{\section}{\sihao\hei}{\thesection}{1em}{}
\titlespacing{\section}{0pt}{18pt  plus 6pt}{18pt  plus 6pt}
% 三级标题
\titleformat{\subsection}{\sihao\hei}{\thesubsection}{1em}{}
\titlespacing{\subsection}{0pt}{12pt plus 3 pt}{12pt plus 3pt}
% 四级标题
\titleformat{\subsubsection}{\xiaosi\hei}{\thesubsubsection}{1em}{}
\titlespacing{\subsubsection}{0pt}{9pt plus 3pt}{9pt plus 3pt}

% 设置目录
\titlecontents{part}[0em]
    {\song\xiaosi}
    {\contentslabel{0em}}
    {}
    {~\titlerule*[0.6pc]{$.$}~\contentspage}
\titlecontents{chapter}[3em]
    {\song\xiaosi}  % 标题格式
    {\contentslabel{3.5em}} % 标题标志 (设置标题标志的格式,如序号格式、序号宽度、序号与标题内容之间的间距等,不可空置)
    {} % 无序号标题 (设置无序号标题的格式,如字体、字体尺寸、位置等。该参数可以空置)
    {~\titlerule*[0.6pc]{$.$}~\contentspage} % 指引线与页码 (设置标题与页码之间的指引线样式以及页码的格式,该参数如果空置,标题将无指引线和页码)
\titlecontents{section}[3em]
    {\song\xiaosi}
    {\contentslabel{1.5em}}
    {\hspace*{-4em}}
    {~\titlerule*[0.6pc]{$.$}~\contentspage}
\titlecontents{subsection}[5em]
    {\song\xiaosi}
    {\contentslabel{2.3em}}
    {\hspace*{-4em}}
    {~\titlerule*[0.6pc]{$.$}~\contentspage}
\titlecontents{subsubsection}[7em]
    {\song\xiaosi}
    {\contentslabel{3.1em}}
    {\hspace*{-4em}}
    {~\titlerule*[0.6pc]{$.$}~\contentspage}

% 设置目录章标题
\ctexset{
    chapter={
    number=\arabic{chapter},
    tocline=\CTEXifname{\protect\numberline{第 \thechapter 章}}{}#2
    }
}
% 设置目录显示深度
\setcounter{tocdepth}{4}


% 设置页眉
\pagestyle{fancy}
\fancyhf{}
% 偶数页页眉
\fancyhead[CO]{\leftmark}
\fancyhead[CE]{\wuhao\song 天津大学硕士学位论文}
\renewcommand{\headrulewidth}{0.1mm} % 设置页眉线粗细

% 页脚
\fancyfoot[C]{\color{gray}\xiaowu \thepage}

% 设置图编号
\renewcommand{\thefigure}{\thechapter-\arabic{figure}}
\captionsetup[figure]{labelsep=space}
% 设置caption格式
\captionsetup{font={normalsize,stretch=2}, justification=raggedright}
% 设置表
\captionsetup[table]{labelsep=space,position=above}
\renewcommand{\thetable}{\thechapter-\arabic{table}}

% 表标题命令
\newcommand{\ctable}[2]{
    \vspace{-2em}
    \caption{#1}
    \vspace{-1.5em}
    \addtocounter{table}{-1}
    \renewcommand{\tablename}{Table.}    
    \caption{#2} 
    \vspace*{-1em}
}

% 图标题命令
\newcommand{\cfigure}[2]{
    \vspace{-2em}
    \caption{#1}
    \vspace{-1em}
    \addtocounter{figure}{-1}
    \renewcommand{\figurename}{Fig.}    
    \caption{#2} 

}

% 重命名参考文献上标
% \biboptions{sort&compress} % 连续引用多篇文献,压缩排序
\newcommand{\ccite}[1]{\textsuperscript{\cite{#1}}}
