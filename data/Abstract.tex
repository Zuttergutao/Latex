% \headrulewidth{0pt}
\pagenumbering{Roman}   % 页码采用罗马数字
\pagestyle{plain}       % 摘要有页脚

% 中文
\begin{center}
    \erhao\song\textbf{摘要}
\end{center}

\song\xiaosi{
% 中文摘要填写 %
%%%%%%%%%%%%%%%%%%%%%%%%%%%%%%%%%%%%%%%%%%%%%%%%%%%%%%%%%%%%%%%%%%%%
以异丁醛废气模拟丁辛醇装置的 VOCs,对生物滴滤塔降解异丁醛的过程进行研究,旨在解决丁辛醇行业的
VOCs 污染问题。结果表明:生物滴滤塔系统的挂膜周期约为 2 周;循环营养液流速在 1.1 ~10.2 mh变化时,对异丁
醛的脱除效率基本无影响;由于鼠李糖脂的存在,当循环营养液流速大于 13.6 mh1 时,生物滴滤塔系统内出现大量
泡沫,导致异丁醛的脱除效率迅速下降;同时,鼠李糖脂也促进生物滴滤塔系统性能的提升,,体现了生物滴滤塔在丁辛醇装置 VOCs 治理中的工业化应用潜力。
%%%%%%%%%%%%%%%%%%%%%%%%%%%%%%%%%%%%%%%%%%%%%%%%%%%%%%%%%%%%%%%%%%%%
}

% 中文关键词
\par
\vspace{1 ex}
\noindent
\song\sihao{\textbf{关键词:}\song\xiaosi{
% 中文摘要关键词填写 %
%%%%%%%%%%%%%%%%%%%%%%%%%%%%%%%%%%%%%%%%%%%%%%%%%%%%%%%%%%%%%%%%%%%%
    关键词1,关键词2,关键词3
%%%%%%%%%%%%%%%%%%%%%%%%%%%%%%%%%%%%%%%%%%%%%%%%%%%%%%%%%%%%%%%%%%%%
}}

\newpage

% 英文
\begin{center}
    \erhao\textbf{ABSTRACT}
\end{center}

% 中文摘要填写 %
%%%%%%%%%%%%%%%%%%%%%%%%%%%%%%%%%%%%%%%%%%%%%%%%%%%%%%%%%%%%%%%%%%%%
The performance of biotrickling filters (BTFs) on VOCs (isobutyraldehyde as a model VOC) removal
from hydroformylation(OXO) units was studied to solve VOCs problems in OXO industry. The results indicate
that approximately two weeks were required for acclimating of microorganisms. The removal efficiency was
constant when the liquid recirculation rate (LRR) varied from 1.1 m·h−1 to 10.2 m·h−1
. Due to the presence of rhamnolipids in the liquid recirculation, LRR over 13.6 m·h−1 led to great foam in BTFs, which results in a
reduction of removal efficiency. Nevertheless, rhamnolipids promote the elimination capacity of BTFs up to
158 gmh at a constant LRR of 4.5 m·h−1 and an empty bed retention time (EBRT) of 40 s. Therefore, BTFs
show industrial application potential on VOCs removal in OXO industry.
%%%%%%%%%%%%%%%%%%%%%%%%%%%%%%%%%%%%%%%%%%%%%%%%%%%%%%%%%%%%%%%%%%%%

% 英文关键字
\par
\vspace{1 ex}
\noindent
\sihao{\textbf{KEY WORDS: }\xiaosi{
% 英文摘要关键词填写 %
%%%%%%%%%%%%%%%%%%%%%%%%%%%%%%%%%%%%%%%%%%%%%%%%%%%%%%%%%%%%%%%%%%%%
    Abstract1, Abstract2, Abstract3
%%%%%%%%%%%%%%%%%%%%%%%%%%%%%%%%%%%%%%%%%%%%%%%%%%%%%%%%%%%%%%%%%%%%
}}


\newpage


